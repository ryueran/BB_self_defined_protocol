\documentclass{article}
\usepackage[utf8]{inputenc}
\usepackage{graphicx}   % 插入图片
\usepackage{amsmath}    % 数学公式
\usepackage{hyperref}   % 超链接

\title{The documentation about communication protocol}
\author{Muzi}
\date{\today}

\begin{document}
\maketitle
\tableofcontents

\section{Design Specification}
\subsection{Key-Value-Store-Protocol}
\subsubsection{Introduction and Overview}
Thid document specifies the Key-Value-Store-Protocol (KVSP). The protocol allows a client to store key-value pairs on a server.
Stored values can be retrieved by their keys. All messages are transferred via a TCP socket. The server listens for incoming connection on such socket.
Clients connect to the server to interact with it.
\subsubsection{Data Types}
All messages consis of values. Each value has one of the following data types:
\begin{tabular}{|l|c|r|}
    \hline
    Data Type & Description\\
    \hline
    uint8 & unsigned 8 bit integer \\
    uint32 & unsigned 32 bit integer \\
    String & see below \\
    Enumeration & see below \\
    \hline
\end{tabular} \\
Note: All values are transmitted in network byte order (big-endian).
\subsection{Enumerations}
They are transferred as uint8 values. \\
\begin{center}
\begin{tabular}{|p{3cm}|p{3cm}|}
    \hline
    \multicolumn{2}{|c|}{MessageType} \\  % {3}表示合并列数,{|c|}为对齐和竖线
    \hline
    Name & Value \\
    Unknown & 0 \\
    Connect & 1 \\
    Accept & 2 \\
    Error & 3 \\
    Store & 4 \\
    Load & 5 \\
    Value & 6 \\
    \hline
\end{tabular}

\begin{tabular}{|p{6cm}|p{3cm}|}
    \hline
    \multicolumn{2}{|c|}{Error} \\  % {3}表示合并列数,{|c|}为对齐和竖线
    \hline
    Name & Value \\
    ConnectMessageExcepted & 0 \\
    KeyNotFound & 1 \\
    \hline
\end{tabular} 
\end{center}\\

\section{Sequence Diagram}
J

\bibliographystyle{plain}
\bibliography{references}
\end{document}